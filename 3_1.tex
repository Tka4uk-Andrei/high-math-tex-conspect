\part{Числовые ряды}
\section{Основные определения}

\begin{Note}[\textcolor{red}{Некоторые случаи рядов}]
    Вещественное число можно представить в следующем виде
    \[
        x \in \bb{R},\; x = A,a_1\,a_2\,\dots\,a_n\,\dots
    \]
    где
    \[
         a_i \in \{0, 1, \dots, 9\}, \quad A \in \bb{Z}
    \]
    Также запись выше эквивалентна следующему выражению
    \[
        x = A + \frac{a_1}{10} + \frac{a_2}{10^2} + \dots + \frac{a_n}{10^n} + \dots 
    \]
    Из этого можно выделить рациональное число  
    \[
        x_n = A,a_1\,a_2\,\dots\,a_n \in \bb{Q}
    \]
    Связь рационального и действительного числа
    \[
        \exists \lim_{n \rightarrow +\infty}x_n = x
    \]
    Рассмотрим следующий случай (вывод убывающей геом. прогрессии). Пусть 
    \[
        b_n = b_1\,q^{n-1} \qquad |q| < 1
    \]
    тогда
    \begin{align*}
         S_n &= b_1 + \dots + b_n\\
         S_n &= b_1 + q\,(b_1 + b_1\,q + \dots + b_1\,q^{n-2})\\
         S_n &= b_1 + q\,(b_1 + b_1\,q + \dots + b_1\,q^{n-2} + b_1\,q^{n - 1}) - b_1\,q^{n}\\
         S_n &= b_1 + q\,S_n - b_1\,q^{n}\\
         S_n\,(1 - q) &= b_1\,(1 - q^{n})\\
         S_n &= b_1\,\frac{1-q^n}{1-q}
    \end{align*}
    Зная что $|q| < 1$, получаем
    \[
        b_1\,\frac{1-q^n}{1-q} \rightarrow \frac{b_1}{1-q} \quad (n \rightarrow + \infty)
    \]
    Значит
    \[
        S_n = \sum^{\infty}_{n = 1}b_1\,q^{n-1} = \frac{b_1}{1-q}
    \]
\end{Note}

\begin{Def}[Числовой ряд]
	Пусть $\{a_n\}$ --- последовательность чисел, $a_n \in \bb{R}$. Тогда формальное выражение 
    \[
        \sum^{+\infty}_{n = 1}\,a_n
    \] 
    называется числовым рядом, где 
    \[
        a_1,\; a_2,\; \dots,\; a_n
    \]
    --- члены ряда, а функция $n \mapsto a_n$ называется общим членом ряда.
\end{Def}

\begin{Def}
	Пусть $a_n$ - числовой ряд. Тогда:
	\begin{enumerate}
        \item n-ная частичная сумма ряда имеет вид
        \[
            S_n = \sum^n_{k = 1}a_k = a_1 + a_2 + \dots + a_n
        \]
         
        \item Если 
        \[
            \exists \lim_{n \rightarrow +\infty} a_n = S
        \]
        то ряд называется сходящимся и $S$ --- его сумма. Пишем 
        \[
            \sum^{\infty}_{n = 1}a_n = S
        \]
         
        \item Если $\nexists\,\lim S_n$, в частности предел равен $\infty$, то ряд расходящийся и такой ряд не имеет суммы.
    \end{enumerate}
\end{Def}

\begin{Def}
	Пусть $\sum^{+\infty}_{n = 1}a_n$ --- числовой ряд. Тогда $\forall n \in \bb{N}$ ряд $\sum^{+\infty}_{k = n + 1}a_k$
    нызывается $n$-ым остатком ряда $\sum^{+\infty}_{n = 1}a_n$
\end{Def}

\pagebreak

\begin{Note}.\\
    \begin{enumerate}
        \item[\textbullet] Если ряд $\sum^{+\infty}_{n = 1}a_n$
        сходится и его сумма равна $S$, то $\forall n \in \bb{N}$ его $n$-ый остаток сходится и его сумма $S_k = S - S_n$.
        
        \item[\textbullet] Если ряд $\sum^{+\infty}_{n = 1}a_n$ расходится, то $\forall n \in N$ его остаток тоже расходится.
    \end{enumerate}
    
\end{Note}

\begin{Note}
    Свойства сходящихся рядов
    \begin{enumerate}
        \item Сумма сходящихся рядов равна сходящимуся ряду  
        \[
            \sum^{+\infty}_{n=1}a_n = S^{(A)} \quad \sum^{+\infty}_{n=1}b_n = S^{(B)}\; \Rightarrow\; \sum^{+\infty}_{n=1}(a_n + b_n) = S^{(A)} + S^{(B)}
        \]
         
        \item При домножении сходящегося ряда на константу ряд остаётся сходящимся
        \[
            \sum^{+\infty}_{n=1}a_n = S^{(A)} \quad \Rightarrow \quad \forall a \in \bb{R} \quad \sum^{+\infty}_{n=1}\alpha\,a_n = \alpha\,S^{(A)}
        \]
    \end{enumerate}
\end{Note}

\begin{Proof} (Пункт 1)
    \begin{gather*}
        S^{(A+B)}_n = S^{(A)}_n + S^{(B)}_n \rightarrow S^{(A)} + S^{(B)}\\
        S^{(A+B)}_n \rightarrow S^{(A+B)}
    \end{gather*}
\end{Proof}

\begin{Proof} (Пункт 2)
    \begin{gather*}
        S^{(\alpha\,A)}_n = \sum^{+\infty}_{n=1}\alpha\,a_n = \alpha\,S^{(A)}_n \rightarrow \alpha\,S^{(A)}\\ S^{(\alpha\,A)} \rightarrow \alpha\,S^{(A)}
    \end{gather*}
\end{Proof}

\begin{Th}[Необходимое условие сходимости]
	Если ряд\\ $\sum^{+\infty}_{n=1}a_n = S$, то $\exists\, \lim_{n \rightarrow +\infty}a_n = 0$
\end{Th}

\begin{Proof}
    Из определения частичной суммы ряда следует
	\[
        a_n = S_n - S_{n - 1} = (a_1 + a_2 + \dots + a_{n-1} + a_n) - (a_1 + a_2 + \dots + a_{n-1})
    \]
    Тогда 
    \[
        \exists \lim_{n \rightarrow +\infty}a_n = \lim_{n \rightarrow +\infty}(S_n - S_{n-1}) = \lim_{n \rightarrow +\infty}S_n - \lim_{n \rightarrow +\infty}S_{n - 1} = S - S = 0
    \]
\end{Proof}

\begin{Note}
	Если $a_n \nrightarrow 0 \quad (n \rightarrow \infty)$, то ряд расходится.
\end{Note}

\textcolor{red}{Не очень понятен пример}
\begin{Example}
    Дано
	\[
        \sum^{+\infty}_{n=1}ln(1 + \frac{1}{n})
    \]
    С одной стороны (через формулы эквивалентности у пределов (1 сем.))
    \begin{gather*}
        a_n = ln(1 + \frac{1}{n}) \sim \frac{1}{n}\\
        a_n \rightarrow 0 \qquad (n \rightarrow \infty)
    \end{gather*}
    С другой
    \begin{align*}
        S_n &= \sum^{n}_{k=1}\ln\left(1 + \frac{1}{k}\right) = \ln\left(\frac{2}{1}\right) + \ln\left(\frac{3}{2}\right) + \dots + \ln\left(\frac{n + 1}{\cancel{n}}\right) =\\
        &= \ln\left(\frac{\cancel{2}}{1} \cdot \frac{\cancel{3}}{\cancel{2}} \dots \frac{n + 1}{\cancel{n}}\right) = \ln(n + 1)
    \end{align*}
    Таким образом
    \[
        S_n = \ln(n + 1) \rightarrow +\infty 
    \]
    значит ряд расходится и тогда 
    \[
        a_n \rightarrow 0 \quad (n \rightarrow +\infty)
    \]
\end{Example}

\begin{Th}[Критерий Коши сходимости числового ряда].\\
	Если ряд $\sum^{+\infty}_{n=1}a_n$ сходится, то
    \[
        \forall \epsilon > 0 \exists N \in N \forall n \geq N \forall p \in N |a_{n+1} + \dots + a_{n+p} < \epsilon|
    \]
    Обратное утверждение также верно
\end{Th}

\begin{Proof}
	$\sum^{+\infty}_{n=1}a_n$ сходится $\Leftrightarrow \exists lim_{n \rightarrow +\infty}S_n = S \in R \Leftrightarrow \forall \epsilon > 0 \exists N \in N \forall n \geq N \forall p \in N |S_{n+p} - S_n| < \epsilon$\\
	$|S_{n+p}-S_n| = |a_{n+1} + ... + a_{n+p}| < \epsilon$\\
\end{Proof}

\begin{Th}[Признак абсолютной сходимости]
	Если ряд $\sum^{+\infty}_{n=1}|a_n|$ сходится, то и ряд $\sum^{+\infty}_{n=1}a_n$ сходится.
\end{Th}

\begin{Proof}
	$\sum^{+\infty}_{n=1}|a_n|$ сходится $\Rightarrow \forall \epsilon > 0 \exists N \in N \forall n \geq N \forall p \geq 1 |a_{n+1}| + ... + |a_{n+p}| < \epsilon \Rightarrow |a_{n+1} + ... + a_{n+p}| \leq |a_{n+1}| + ... + |a_{n+p}| < \epsilon$\\
	По критерию Коши ряд сходится.\\
\end{Proof}

\begin{Def}
	Если ряд $\sum^{+\infty}_{n=1}|a_n|$ сходится, ряд $\sum a_n$ называется абсолютно сходящимся.
\end{Def}