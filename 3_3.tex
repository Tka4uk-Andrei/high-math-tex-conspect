\section{Сходимость знакопеременных рядов}

\begin{Def}
    Ряд $\sum_{n=1}^{+\infty}a_n$ наз. знакопеременным, если существует беск. много $n \in \mathbb{N} a_n>0$ 
 и существует беск. много $n\in \mathbb{N} a_n<0$\\
\end{Def}

\begin{Def}
    Ряд $\sum (-1)^{n-1} a_n = a_1-a_2+a_3-a_4+...$ наз. знакочередующимся, если $\forall n \in \mathbb{N} a_n>0$\\
\end{Def}

\begin{Def}[признак Лейбница сходимость знакоперем. рядов]
    Пусть ряд $\sum_{n=1}^{+\infty} (-1)^{n-1} a_n $ удовл. след. условиям:\\
    1)$\forall n \in \mathbb{N}, a_n>0$\\
    2)$\{a_n\} \searrow$ строго т.е. $\forall n \in \mathbb{N}, a_{n+1}<a_n$\\
    3)$\exists lim_{n->\infty}=0$\\
    Тогда ряд $\sum_{n=1}^{+\infty} (-1)^{n-1} a_n = S$ сходится, S - его сумма, при этом\\
    1)$a_1-a_2<S<a_1$\\
    2)$\forall n \geqslant 1 |S-S_n|<a_{n+1}$\\
\end{Def}

\begin{Proof}
    Рассмотрим $\{S_{2n}\}$\\
    $S_{2n}=\underbrace{(a_1-a_2)}_{>0}+\underbrace{(a_3-a_4)}_{>0}+...+\underbrace{(a{2n-1}-a_{2n})}_{>0}$\\
    $S_{2(n+1)}-S_{2n}=a_{2n+1}-a_{2n+2}>0 \forall n \geqslant 1, S_{2n}<S_{2(n+1)}=>\{S_{2n}\} \searrow$ строго\\
    Теперь рассмотрим другую группировку\\
    $S_{2n}=\underbrace{a_1}_{>0}-\underbrace{(a_2-a_3)}_{>0}-...-\underbrace{(a_{2n-2}-a_{2n-1})}_{>0}-\underbrace{a_{2n}}_{>0}$\\
    =>$S_{2n}<a_1(n\geqslant 1), S_{2n}<a_1-a_2+a_3(n\geqslant 2)$\\
    $S_{2n}$ огр. сверху\\
    =>$\exists lim_{n->\infty}S_{2n}=S=sup_{n\geqslant 1} S_{2n}\geqslant a_1-(a_2-a_3)<a_1$\\
    =>$\exists lim_{n->\infty}S_{2n+1}= lim_{n->\infty}(\underbrace{S_{2n}}_{->S}+\underbrace{a_{2n+1}}_{->0})=S$\\
    =>$\exists lim_{n->\infty}S_{n}=S<a_1$\\
    $\sum_{n=1}^{+\infty} (-1)^{n-1} a_n=a_2-a_3+a_4-...=a_1-S<a_2$\\
    =>$a_1-a_2<S$(пункт 1 доказан)\\
    Второй пункт доказ. аналогично\\
    $(-1)^n\sum_{k=n+1}^{+\infty} (-1)^{k-1} a_k=a_{n+1}-a_{n+2}+...=\underbrace{(-1)^n(S-S_n)}_{0<a_{n+1}-a_{n+2}<}$\\
    => $|S-S_n|<a_{n+1}$\\
\end{Proof}

\begin{Example}
    Доказ. $\sum_{n=1}^{+\infty} \frac{(-1)^{n-1}}{n}$ сходится(не абсолютно)\\
    $\frac{(-1)^{n-1}}{n}$ - ряд Лейбница\\
    Решение:\\
    $a_n=\frac{1}{n}>0, {a_n}\searrow$ строго, $a_n->0 (n->\infty)$ => по т.1 ряд Лейбница сходится\\
    Проверка на абс. сход.\\
    $\sum_{n=1}^{+\infty}| \frac{(-1)^{n-1}}{n}|=\sum_{n=1}^{+\infty}\frac{1}{n}$расход. (гарм. ряд)\\
\end{Example}


\begin{Example}
	$\sum^{\infty}_{n=1} \frac{(-1)^{n-1}}{n^2}$ - сходится абсолютно\\
	$\sum^{\infty}_{n=1} \frac{(-1)^{n-1}}{ln(n+1)}$ - сходится (не абсолютно)\\
\end{Example}

\begin{Th}[Признак Дирихле]
	Пусть $\sum^{infty}_{n=1}a_nb_n$ - числовой ряд, удовлетворяющий следующим условиям:\\
	1) $\exists M > 0 : \forall n \in N |\sum^n_{k=1}a_k| \leq M$, то есть частичная сумма ряда ограничена\\
	2) $\{b_n\} \rightarrow 0$, то есть $\forall n \geq 1, b_k > 0;$ $\forall n \geq 1 b_{n+1} < b_n;$ $\exists lim b_n = 0$\\
	Тогда ряд $\sum^{infty}_{n=1}a_nb_n$ сходится и его сумма $|T| \leq Mb_1$\\
\end{Th}

\begin{Proof}
	$T_n = a_1b_1 + ... + a_nb_n = (a_1b_1) + (a_1b_2 + a_2b_2 - a_1b_2) + (a_1b_3 + a_2b_3 + a_3b_3 - a_1b_3 - a_2b_3) + ... + a_1b_n + a_2b_n + ... + a_nb_n - a_1b_n - a_2 b_n - ... - a_{n_1}b_n = $\\
	$ = S_1b_1 + S_2b_2 + ... + S_nb_n - S_1b_2 - S_2b_3 - ... - S_{n_1}b_n = S_1(b_1-b_2) + S_2(b_2 - b_3) + ... + S_{n-1}(b_{n-1} - b_n) + S_nb_n$\\
	$\sum^{infty}_{k=1}a_kb_k = \sum^{\infty}_{k=1} S_k(b_k - b_{k+1})$\\
	$|S_k(b_k - b_{k+1})| < M(b_k - b_{k+1})$\\
	$\sum^{n}_{k=1}a_kb_k \leq \sum^{n}_{k=1} |S_k(b_k - b_{k+1})| < M\sum^{n}_{k=1}(b_k - b_{k+1}) = M(b_1 - b_n) \rightarrow Mb_1$\\
	По признаку сравнения ряд $\sum^{+\infty}_{n=1}S_k(b_k - b_{k+1}) = T$ - сходится\\
	$T_n = \sum^{n-1}_{k=1} S_k(b_k - b_{k+1}) + S_nb_n \rightarrow T \Rightarrow \exists limT_n = T$\\
	Ч.т.д\\
\end{Proof}

\begin{Note}
	$\sum (-1)^{n-1}b_n, b_n \rightarrow 0$\\
	$a_n = (-1)^{n-1}, |S_n| \leq 1$
\end{Note}

\begin{Example}
	$\sum^{+\infty}_{n=1} \frac{cos n\alpha}{n^{\beta}}$\\
	$1) \alpha = 2\Pi k, k \in Z$ $cos(2 \Pi nk) = 1 \Rightarrow \sum^{\infty}_{n=1} \frac{1}{n^{\beta}}, \beta \leq 0 \Rightarrow \frac{1}{n^{\beta}}$ расходится, иначе сходится\\
	$2) \alpha \neq 2 \Pi k$\\
	$a_n = cos n\alpha, b_n = \frac{1}{n^{\beta}} \rightarrow 0$ при $\beta > 0$\\
	$S_n = cos \alpha + ... cos n\alpha$\\
	$2cos\alpha S_n = 2cos^2\alpha + 2cos\alpha cos2\alpha + ... + 2cos\alpha cosn\alpha = 1 + cos2\alpha + cos\alpha + cos3\alpha + ... + cos(n-1)\alpha + cos(n+1)\alpha = 1 + S_n - cosn\alpha + S_n - cos \alpha + cos(n+1)\alpha$\\
	$2(1-cos\alpha)S_n = 1 + cos n\alpha + cos\alpha - cos(n+1)\alpha$\\
	$2(1-cos\alpha)|S_n| = 4 \Rightarrow |S_n| \leq \frac{2}{1-cos\alpha} = M \Rightarrow$ ряд сходится\\
\end{Example}

