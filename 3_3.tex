\section{Сходимость знакопеременных рядов}

\begin{Def}~\\
    Ряд $\sum_{n=1}^{+\infty}a_n$ называют знакопеременным, если существует бесконечно много $n \in \bb{N} a_n>0$
    и бесконечно много $k \in \mathbb{N} a_k<0$\\
\end{Def}

\begin{Def}~\\
    Ряд 
    \[
        \sum (-1)^{n-1} a_n = a_1 - a_2 + a_3 - a_4 + \dots
    \] 
    называют знакочередующимся, если $\forall n \in \mathbb{N} \quad a_n > 0 $
\end{Def}

\begin{Th}[признак Лейбница сходимость знакоперем. рядов]~\\
    Пусть ряд $\sum_{n=1}^{+\infty} (-1)^{n-1} a_n $ удовлетворяет следующим условиям:
    \begin{enumerate}
        \item $\forall n \in \mathbb{N}, \quad a_n > 0$
         
        \item $\{a_n\} \searrow$ строго, то есть $\forall n \in \bb{N}, \quad a_{n+1}<a_n$
        
        \item $\exists \lim\limits_{n \to \infty}=0$
        
    \end{enumerate}
    Тогда ряд $\sum_{n=1}^{+\infty} (-1)^{n-1} a_n = S$ сходится, $S$ - его сумма, при этом
    \begin{enumerate}
        \item $a_1-a_2<S<a_1$
        
        \item $\forall n \geqslant 1 \quad |S-S_n|<a_{n+1}$
    \end{enumerate}
\end{Th}

\begin{Proof}~\\
    Рассмотрим $\{S_{2n}\}$ тогда
    \[
        S_{2\,n}=\underbrace{(a_1-a_2)}_{>0}+\underbrace{(a_3-a_4)}_{>0}+\dots+\underbrace{(a_{2\,n-1}-a_{2\,n})}_{>0} 
    \]
    Также заметим
    \[
        S_{2\,(n+1)} - S_{2\,n}= a_{2\,n+1}-a_{2\,n+2} > 0 \quad \forall n \geqslant 1
    \]
    Так как по условию $\{a_n\}$ возрастает строго, тогда $S_{2n}<S_{2(n+1)}$ и $\{S_{2n}\}$ строго возрастает.\\
    Теперь рассмотрим другую группировку.
    \[
        S_{2n}=\underbrace{a_1}_{>0}-\underbrace{(a_2-a_3)}_{>0}-\dots-\underbrace{(a_{2n-2}-a_{2n-1})}_{>0}-\underbrace{a_{2n}}_{>0}
    \]
    Из этого следует 
    \[
        S_{2\,n}<a_1 \quad (n \geqslant 1) \qquad S_{2\,n}<a_1-a_2+a_3(n \geqslant 2)
    \]
    Значит $S_{2n}$ ограниченна сверху. Тогда
    \begin{gather*}
        \exists \lim\limits_{n \to \infty} S_{2\,n} = S = \sup\limits_{n\geqslant 1} S_{2\,n} \leqslant a_1-(a_2-a_3) < a_1
    \end{gather*}
    Значит
    \begin{gather*}
        \exists \lim\limits_{n \to \infty}S_{2\,n+1}= 
        \lim\limits_{n\to\infty}(\underbrace{S_{2\,n}}_{\to S}+\underbrace{a_{2n+1}}_{\to 0})=S\\
        \exists \lim\limits_{n \to \infty}S_{n}=S<a_1
    \end{gather*}
    Таким образом
    \[
        \sum_{n=2}^{+\infty} (-1)^{n-1} a_n=a_2-a_3+a_4-\dots=a_1-S<a_2
    \]
    Значит $a_1-a_2<S$ (пункт 1 доказан)\\
    Второй пункт доказывается аналогично\\
    \[
        (-1)^n\sum_{k=n+1}^{+\infty} (-1)^{k-1} a_k=a_{n+1}-a_{n+2} + \dots =\underbrace{(-1)^n(S-S_n)}_{0 < a_{n+1}-a_{n+2} < a_{n + 1}}
    \]
    Таким образом $|S-S_n| < a_{n+1}$
\end{Proof}

\textcolor{red}{не разобрано}
\begin{Example}~\\
    Доказ. $\sum_{n=1}^{+\infty} \frac{(-1)^{n-1}}{n}$ сходится(не абсолютно)\\
    $\frac{(-1)^{n-1}}{n}$ - ряд Лейбница\\
    Решение:\\
    $a_n=\frac{1}{n}>0, {a_n}\searrow$ строго, $a_n->0 (n->\infty)$ => по т.1 ряд Лейбница сходится\\
    Проверка на абс. сход.\\
    $\sum_{n=1}^{+\infty}| \frac{(-1)^{n-1}}{n}|=\sum_{n=1}^{+\infty}\frac{1}{n}$расход. (гарм. ряд)\\
\end{Example}

\textcolor{red}{не разобрано}
\begin{Example}~\\
	$\sum^{\infty}_{n=1} \frac{(-1)^{n-1}}{n^2}$ - сходится абсолютно\\
	$\sum^{\infty}_{n=1} \frac{(-1)^{n-1}}{ln(n+1)}$ - сходится (не абсолютно)\\
\end{Example}

\textcolor{red}{не разобрано}
\begin{Th}[Признак Дирихле]~\\
	Пусть $\sum^{infty}_{n=1}a_n\,b_n$ - числовой ряд, удовлетворяющий следующим условиям:
    \begin{enumerate}
        \item $\exists M > 0 : \forall n \in N |\sum^n_{k=1}a_k| \leq M$, то есть частичная сумма ряда ограничена
        
        \item $\{b_n\} \rightarrow 0$, то есть $\forall n \geq 1, b_k > 0;$ $\forall n \geq 1 b_{n+1} < b_n;$ $\exists lim b_n = 0$\\
        Тогда ряд $\sum^{infty}_{n=1}a_nb_n$ сходится и его сумма $|T| \leq Mb_1$
        
    \end{enumerate}
\end{Th}

\textcolor{red}{не разобрано}
\begin{Proof}~\\
	$T_n = a_1\,b_1 + \dots + a_n\,b_n = (a_1\,b_1) + (a_1\,b_2 + a_2\,b_2 - a_1\,b_2) + (a_1\,b_3 + a_2\,b_3 + a_3\,b_3 - a_1\,b_3 - a_2\,b_3) + \dots + a_1\,b_n + a_2\,b_n + \dots + a_n\,b_n - a_1\,b_n - a_2\,b_n - \dots - a_{n_1}\,b_n = $\\
	$ = S_1b_1 + S_2b_2 + \dots + S_nb_n - S_1b_2 - S_2b_3 - \dots - S_{n_1}b_n = S_1(b_1-b_2) + S_2(b_2 - b_3) + \dots + S_{n-1}(b_{n-1} - b_n) + S_nb_n$\\
	$\sum^{infty}_{k=1}a_kb_k = \sum^{\infty}_{k=1} S_k(b_k - b_{k+1})$\\
	$|S_k(b_k - b_{k+1})| < M(b_k - b_{k+1})$\\
	$\sum^{n}_{k=1}a_kb_k \leq \sum^{n}_{k=1} |S_k(b_k - b_{k+1})| < M\sum^{n}_{k=1}(b_k - b_{k+1}) = M(b_1 - b_n) \rightarrow Mb_1$\\
	По признаку сравнения ряд $\sum^{+\infty}_{n=1}S_k(b_k - b_{k+1}) = T$ - сходится\\
	$T_n = \sum^{n-1}_{k=1} S_k(b_k - b_{k+1}) + S_nb_n \rightarrow T \Rightarrow \exists limT_n = T$
\end{Proof}

\begin{Note}~\\
	Если
    \[
        \sum_{n=1}^{+\infty}(-1)^{n-1}b_n, \qquad b_n \rightarrow 0
    \]
    То значение суммы можно свести к следующему
	\[
        a_n = (-1)^{n-1}, |S_n| \leqslant 1    
    \]
\end{Note}

\textcolor{red}{Не разобран}
\begin{Example}~\\
	$\sum^{+\infty}_{n=1} \frac{cos n\alpha}{n^{\beta}}$\\
	$1) \alpha = 2\Pi k, k \in Z$ $cos(2 \Pi nk) = 1 \Rightarrow \sum^{\infty}_{n=1} \frac{1}{n^{\beta}}, \beta \leq 0 \Rightarrow \frac{1}{n^{\beta}}$ расходится, иначе сходится\\
	$2) \alpha \neq 2 \Pi k$\\
	$a_n = cos n\alpha, b_n = \frac{1}{n^{\beta}} \rightarrow 0$ при $\beta > 0$\\
	$S_n = cos \alpha + ... cos n\alpha$\\
	$2cos\alpha S_n = 2cos^2\alpha + 2cos\alpha cos2\alpha + ... + 2cos\alpha cosn\alpha = 1 + cos2\alpha + cos\alpha + cos3\alpha + ... + cos(n-1)\alpha + cos(n+1)\alpha = 1 + S_n - cosn\alpha + S_n - cos \alpha + cos(n+1)\alpha$\\
	$2(1-cos\alpha)S_n = 1 + cos n\alpha + cos\alpha - cos(n+1)\alpha$\\
	$2(1-cos\alpha)|S_n| = 4 \Rightarrow |S_n| \leq \frac{2}{1-cos\alpha} = M \Rightarrow$ ряд сходится\\
\end{Example}

