\section{Разложение функции в ряд Тейлора}

\newcommand*\circled[1]{\tikz[baseline=(char.base)]{
		\node[shape=circle,draw,inner sep=2pt] (char) {#1};}}

\begin{Th}(О разложении функции в степенной ряд)\\
	Пусть функция $f(x) = \sum\limits_{n=0}^{+\infty}c_n(x-a)^n, x \in U_{\delta}(a), \delta > 0$\\
	Тогда такое разложение единственное и его коэффициенты вычисляются по формуле: 
	$\begin{cases}
	  c_0 = f(a)\\
	  \forall n \in \bb{N} \quad c_n =\frac{f(a)^{(n)}}{n!}
	\end{cases}$, где $a$ - фиксированная точка
\end{Th}

\begin{Proof}\\
	$f(x)$ - сумма степенного ряда $\Rightarrow f(x) \in C^{\infty}(U_{\delta}(a))$
	\begin{enumerate}[noitemsep]
		\item $f(x) = c_0 + c_1(x-a) + \dots + c_n(x-a)^n + \dots$ при $x=a: f(a) = c_0$
		\item Продифференциируем: $f'(x) = c_1 + 2c_2(x-a) + \dots + nc_n(x-a)^{n-1} + \dots$\\
			$x=a: f'(a) = c_1$
		\item[\vdots]
		\item[$n)$]$f^{(n)}(x) = n!c_n + \frac{(n+1)!}{1!}c_{n+1}(x-a) + \dots + \frac{(n+k)!}{k!}c_{n+k}(x-a)^k + \dots\\
			x = a: f^{(n)}(a) = n!c_n \Rightarrow c_n = \frac{f^{(n)}(a)}{n!} \: (n \in \bb{N})$
	\end{enumerate}
\end{Proof}

\begin{Def}
	Пусть функция $f(x) \in C^{\infty}(U_{\delta}(a))$\\
	Тогда ряд $\sum\limits_{n=0}^{+\infty}\frac{f^{(n)}(a)}{n!}(x-a)^n$ называется рядом Тейлора функции $f(x)$ в $(\cdot) a$. 
	$$f(x) \to \sum\limits_{n=0}^{+\infty}\frac{f^{(n)}(a)}{n!}(x-a)^n$$
\end{Def}

\begin{Note}~\\
	$f(x) = 
	\begin{cases}
		e^{-\frac{1}{x^2}}, x \neq 0\\
		0, x = 0
	\end{cases}, f(x) \in C^{\infty}(\bb{R}); f'(x) = 
	\begin{cases}
		\frac{2}{x^3}e^{-\frac{1}{x^2}}, x \neq 0\\
		0, x = 0
	\end{cases}$\\
	$\forall k \in \bb{N} \quad f^{(k)}(0) = 0 \Rightarrow$ нулевой ряд Тейлора в окрестности нуля\\
	$\sum\limits_{n=0}^{+\infty}\frac{f^{(n)}(0)}{n!}x^n = 0, x \in \bb{R} \neq f(x) \text{ при } x \neq 0$
\end{Note}

\begin{Note}
	$f(x) \in C^{n}(U_{\delta}(a)) \text{ и } \exists f^{(n+1)} \text{ для } x \in U_{\delta}(a) \Rightarrow\\
	\Rightarrow \forall x \in U_{\delta}(a) \; \exists \xi \in U_{\delta}(a), \xi \in [a;x]\\
	f(x) = \sum\limits_{k=0}^{n}\frac{f^{(k)}(0)}{k!}x^k + R_n(x); R_n(x) = f(x) - \sum\limits_{k=0}^{n}\frac{f^{(k)}(0)}{k!}x^k$\\
	$R_n(x) = \frac{f^{(n+1)}(\xi)}{(n+1)!}(x-a)^{n+1}$ - остаточный член в формуле Тейлора (в данном случае остаточный член представлен в форме Лагранжа)
\end{Note}

\begin{Th}(Критерий разложимости функции в ряд Тейлора)
	Пусть функция $f(x) \in C^{\infty}(U_{\delta}(a)), R > 0$\\
	Тогда $f(x) = \sum\limits_{n=0}^{+\infty}\frac{f^{(n)}(a)}{n!}(x-a)^n, x \in (a-R;a+R) \Leftrightarrow \\
	\Leftrightarrow \forall x \in (a-R;a+R) \exists \lim\limits_{n \to +\infty}R_n(x) = 0$
\end{Th}

\begin{Proof}
	\begin{itemize}
		\item[\circled{$\Rightarrow$}] ~\\
			Пусть $f(x) = 	\sum\limits_{n=0}^{+\infty}\frac{f^{(n)}(a)}{n!}(x-a)^n \Rightarrow\\
			\Rightarrow \forall x \in (a-R;a+R) \; \exists \lim\limits_{n \to +\infty}\sum\limits_{k=0}^{n}\frac{f^{(k)}(0)}{k!}x^k = f(x) \Rightarrow\\
			\Rightarrow \exists \lim\lim\limits_{n \to +\infty}(f(x) - \sum\limits_{k=0}^{n}\frac{f^{(k)}(0)}{k!}x^k) = 0 \Rightarrow\\
			\Rightarrow \exists \lim\lim\limits_{n \to +\infty}R_n(x) = 0 (x \in (a-R;a+R))$
		\item[\circled{$\Leftarrow$}] ~\\
			$\forall x \in (a-R;a+R) \; \exists \lim\limits_{n \to +\infty}R_n(x) = 0 \; \Rightarrow \exists \lim\limits_{n \to +\infty}(f(x) - \sum\limits_{k=0}^{n}\frac{f^{(k)}(0)}{k!}x^k) =\\
			= \lim\limits_{n \to +\infty}f(x) - \lim\limits_{n \to +\infty}\sum\limits_{k=0}^{n}\frac{f^{(k)}(0)}{k!}x^k = f(x)  - \sum\limits_{n=0}^{+\infty}\frac{f^{(n)}(a)}{n!}(x-a)^n = 0 \Rightarrow\\
			\Rightarrow \forall x \in (a-R;a+R) f(x) = \sum\limits_{n=0}^{+\infty}\frac{f^{(n)}(a)}{n!}(x-a)^n$
	\end{itemize}
\end{Proof}

\begin{Note}(Таблица разложений)
	\begin{enumerate}
		\item $e^x = \sum\limits_{n=0}^{+\infty}\frac{x^n}{n!}, \quad R = +\infty$
		\item $\cos(x) = \sum\limits_{n=0}^{+\infty}\frac{(-1)^n x^{2n}}{(2n)!}, \quad R = +\infty$
		\item $\sin(x) = \sum\limits_{n=0}^{+\infty}\frac{(-1)^n x^{2n+1}}{(2n+1)!}, \quad R = +\infty$
		\item $(1+x)^{\mu} = 1 + \sum\limits_{n=1}^{+\infty}\frac{\mu(\mu - 1)(\mu - 2) \dots (\mu - n + 1)}{n!}x^n, \quad R = 1, \mu \in \bb{R} \backslash (\bb{N} \cup {0})$
		\item $\frac{1}{1-x} = \sum\limits_{n=0}^{+\infty}x^n, \quad R = 1$
		\item $\frac{1}{\sqrt{1-x}} = 1 + \sum\limits_{n=1}^{+\infty}\frac{(2n-1)!!}{(2n)!!}x^n, \quad R = 1\\
		(2n-1)!! = 1 \cdot 3 \cdot 5 \cdot \dots \cdot (2n-1), (2n)!! = 2 \cdot 4 \cdot 6 \cdot \dots \cdot 2n$
		\item $\ln(1+x) = 1 + \sum\limits_{n=1}^{+\infty}\frac{(-1)^n x^{2n+1}}{n}, \quad R = 1$
		\item $\arctan(x) = \sum\limits_{n=0}^{+\infty}\frac{(-1)^n x^{2n+1}}{(2n+1)!}, \quad R = 1$
		\item $\arcsin(x) = x + \sum\limits_{n=0}^{+\infty}\frac{(2n-1)!!}{(2n)!!} \cdot \frac{x^{2n+1}}{2n+1}, \quad R = 1$
	\end{enumerate}
\end{Note}

\begin{Example}~\\
	$f(x) = \frac{x}{x^2+x-2} = \sum\limits_{n=0}^{+\infty}c_n x^n, c_n = ?, x \in ?$
	$\frac{x}{x^2+x-2} = \frac{A}{x - 1} + \frac{B}{x + 2} = \frac{(A+B)x + (2A-B)}{x^2+x-2}$\\
	$\begin{cases}
		A + B = 1\\
		2A - B = 0
	\end{cases} \Rightarrow
	\begin{cases}
		A = \frac{1}{3}\\
		B = \frac{2}{3}
	\end{cases} \: f(x) = \frac{1}{3(x-1)} + \frac{2}{3(x+2)} = -\frac{1}{3}\sum\limits_{n = 0}^{+\infty} x^n + \frac{1}{3}\sum\limits_{n = 0}^{+\infty}\frac{(-1)^n x^n}{2^n} =\\
	= \sum\limits_{n = 0}^{+\infty}\frac{1}{3}(\frac{(-1)^n}{2^n} - 1)x^n = \sum\limits_{n = 0}^{+\infty}c_n x^n \Rightarrow c_n = \frac{1}{3}(\frac{(-1)^n}{2^n} - 1) \Rightarrow R = 1$
\end{Example}
