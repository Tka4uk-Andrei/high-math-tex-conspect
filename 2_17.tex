\section{Система линейных ДУ с const коэф.}

\begin{Def}
    Система вида
    \[
    \begin{cases}
        y_1'= a_{11}\,y_1 + \dots + a_{1n}\,y_n + f_1(x)\\
        \vdots\\
        y_n' = a_{n1}\,y_1 + \dots + a_{nn}\,y_n + f_n(x)
    \end{cases}
    \]
    называется системой из $n$ ЛДУ с постоянными коэфинциентами.\\
    \[
        A=
        \begin{pmatrix} 
            a_{11}& \dots & a_{1n}\\  
            \dots& \dots & \dots\\ 
            a_{n1}& \dots & a_{nn}
        \end{pmatrix}
    \]
    --- матрица коэфициентов
\end{Def}

\begin{Note}~\\
    Система из определения 1 сводится для каждой функции $y_i'(x)$ к ЛДУ n-го порядка
    \begin{gather*}
        \vec{y} = \begin{pmatrix} y_{1}\\ \vdots\\ y_{n}\end{pmatrix} = e^{\alpha x}\, \begin{pmatrix} b_{1}\\ \vdots\\ b_{n} \end{pmatrix} \qquad
        \vec{y'} = \alpha\,\vec{b}\,e^{\alpha\,x} \qquad \alpha\,\vec{b}\,e^{\alpha\,x} = A\,\vec{b}\,e^{\alpha\,x}\\
        \Rightarrow \quad A\,\vec{b}=\alpha\,\vec{b}
    \end{gather*}
    Таким образом $\vec{b}$ собственный вектор, $\alpha$ собственное значение матрицы $A$,  $\det(A-\alpha\,E)=0$\\
    \[
        (-1)^n\,det(A-\lambda\, E) = P_n(\lambda)=\lambda^n + p_1\,\lambda^{n-1} + \dots + p_n
    \]
    --- характеристический многочлен.
\end{Note}

\begin{Note}~\\
    Если $\alpha_1\;\dots\;\alpha_n \in \bb{R}$ --- попарно различные корни характеристического многочленана (т.е. собственное значение матрицы $A$)
    $\vec{b_1},\; \dots,\; \vec{b_n}$ соотвествующие собственные векторы. Тогда 
    \[
        y=C_1\,\vec{b_1}\,e^{\alpha_1\,x} + \dots + C_n\,\vec{b_n}\,e^{\alpha_n\,x}
    \]
    --- общее решение линейной диференциальной системой $\vec{y'}=A\,\vec{y}$
\end{Note}

\textcolor{red}{Пример ниже не разобран. Есть множество вопросов по решению. Не понятен ход действий.}
\begin{Example}~\\
    Дано
    \begin{gather*}
        y_1=y_1(t), \quad y_2=y_2(t)\\
        \begin{cases}
            y_1' = y_1 - y_2 + 3\,t\\
            y_2' = -4\,y_1 + y_2 + t\,e^{-t}
        \end{cases}
    \end{gather*}
    Решение.\\
    Сначала рассматриваем соотвествующую однородную систему. Получаем следующий характеристический многочлен.
    \[
        P(\lambda)=\begin{vmatrix} 1-\lambda & -1\\ -4 & 1-\lambda\end{vmatrix}=(\lambda-1)^2-4=(\lambda+1)(\lambda-3)
    \]
    Его решения с кратностями
    \[
        \alpha_1 = -1, \quad \alpha_2 = 3, \quad k_{1,2} = 1
    \]
    Ищем собственный вектор для $\lambda=\alpha_1=-1$. Получаем
    \[
        \begin{pmatrix} 2 & -1\\ -4 & 2\end{pmatrix}\sim \begin{pmatrix} 2 & -1\\ 0 & 0\end{pmatrix} \quad \Leftrightarrow \quad 
        \begin{cases}
            2\,\alpha_1 - \alpha_2 = 0\\
            0\,\alpha_1 + 0\,\alpha_2 = 0
        \end{cases}
        \quad 2\,\alpha = \beta
    \]
    Таким образом собственный вектор
    \[
    \vec{b_1}= \begin{pmatrix} 1\\ 2 \end{pmatrix}
    \]
    Частное решение
    \[ 
        \vec{\phi_1} = \begin{pmatrix} 1 \\ 2 \end{pmatrix}\,e^{-t}
    \]
    Аналогично для $\lambda=\alpha_2=3$
    \[
        \begin{pmatrix} -2 & -1\\ -4 & -2\end{pmatrix} \sim \begin{pmatrix} -2 & -1 \\ 0 & 0\end{pmatrix} \Rightarrow -2\beta_1 = \beta_2
    \]
    Значение собственного вектора и частное решение
    \[
        \vec{b_2}=\begin{pmatrix} 1 \\ -2 \end{pmatrix} \qquad \vec{\phi_2}=\begin{pmatrix} 1 \\  -2 \end{pmatrix}\,e^{3t}
    \]
    Таким образом решение для однородной системы
    \[
        \vec{y_{\text{о.о.}}}=C_1\begin{pmatrix} 1\\ 2\end{pmatrix}\,e^{-t} + C_2\begin{pmatrix} 1\\ - 2\end{pmatrix}\,e^{3\,t}
    \]
    \begin{multline*}
            y_1'' = y_1'- y_2' + 3 = y_1' + 4\,y_2 - y_2 - t\,e^{-t} + 3 = \\ 
            = y_1' + 4\,y_1 - y_1 + y_1' - 3\,t + 3 - t\,e^{-t}
    \end{multline*}
    Для нахождения значения $ y_1''$ использовали следующие равенства
    \begin{align*}
        y_2&= y_1 - y_1' + 3\,t \quad \Leftrightarrow \quad y_1' = y_1 - y_2 + 3\,t\\
        y_2'&= -4\,y_1 + y_2 + t\,e^{-t}
    \end{align*}
    Продолжаем дальше
    \begin{align*}
        L(y_1)&=y_1''-2y_1'-3y_1=3-3t-te^{-t}\\
        P_2(\lambda)&=\lambda^2-2\lambda-3(\lambda+1)(\lambda-3)\\
        L(y)&=3-3t=Q_m(t)e^{\alpha t}, \quad m =1,\alpha=0,k=0\\
        \Psi_1&=At+Bs\\
        \Psi_1'&=A\\
        \Psi_1&=t-\frac{5}{3}
    \end{align*}
    \begin{gather*}
        \begin{vmatrix} 
            \lambda^2 - \lambda - 3 \\ 2\,\lambda-2 \end{vmatrix} \begin{matrix}
            = 3 \\ = 2
        \end{matrix}
    \end{gather*}
    \begin{align*}
        L(\Psi_1)=-3At-3B-2A=3-3t, A=1, B=-\frac{5}{3}\\
        L(y)=-t\,e^{-t},\; m=1,\; \alpha=-1,\; k=1\\
        \Psi_2 = (A\,t^2+B\,t)\,e^{-t} = u\,e^{-t} \qquad \text{где} \quad u=A\,t^2+B\,t\\
        u'=2\,A\,t+B \quad u''= 2\,A \quad \Rightarrow \quad \lambda^2-\lambda-3 \quad (0)
    \end{align*}
    $=>;2\lambda-2(-4);(1)$\\
$L(\Psi_2)=e^{-t}(-8At-4B+2A)=te^{-t}=>$\\
$-8A=1$=>$A=\frac{-1}{8}$\\
$2A-4B=0$=>$B=\frac{1}{16}$\\
$\Psi_2=frac{1}{16}(2t^2+t)e^{-t}$\\
$y_1=c_1e^{-t}+c_2e^{3t}+t-\frac{5}{3}-\frac{1}{16}(2t^2+t)e^{-t}$\\
$y_1'=-c_1e^{-t}+3c_2e^{3t}+1-\frac{1}{16}(-2t+t+1)e^{-t}$\\
$y_2=y_1-y'_1+3t$\\
    \[
    \begin{cases}
        y_2=2c_1e^{-t}-2ce^{3t}+4t-\frac{8}{3}-\frac{1}{16}(4t^2-2t+1)e^{-t}\\
        y_1=c_1e^{-t}+c_2e^{3t}+t-\frac{5}{3}-\frac{1}{16}(2t^2+t)e^{-t}
        \end{cases}
    \]
\end{Example}
