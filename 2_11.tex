\author{Tkachuk Andrei}

\section{Метод вариации произвольных постоянных}

\begin{Th}
    Пусть 
    \[
        p_1(x),\; \dots,\; p_n(x),\; f(x) \in C_{(a;\,b)}
    \]
    и изветна ФСР $\{\varphi_1(x),\, \dots\,\, \varphi_n(x)\}$ ЛОДУ
    \[
        L(y) = y^{(n)} + y^{(n-1)}\,p_1(x) + \dots + y\,p_n(x) = 0
    \]
    Ищем решение ЛНДУ $L(x) = f(x)$ в виде 
    \[
        y = C_1(x)\,\phi_1(x) + \dots + C_n(x)\,\phi_n(x)
    \]
    где функции $\{C_1(x),\, \dots,\, C_n(x)\}$ диференцируемые на $(a;\, b)$
    и $\{C'_1(x),\, \dots,\, C'_n(x)\}$ удовлетворяет следующим свойствам алгебраических уравнений
    \[
        \begin{cases}
            C'_1\,\phi'_1(x) + \dots + C'_n\,\phi'_n(x) = 0\\
            C'_1\,\phi'_1(x) + \dots + C'_n\,\phi'_n(x) = 0\\
            \vdots
            C'_1\,\phi^{(n-2)}_1(x) + \dots + C_n\,\phi^{(n-2)}_n(x) = 0\\
            C'_1\,\phi^{(n-1)}_1(x) + \dots + C_n\,\phi^{(n-1)}_n(x) = f(x)\\
        \end{cases}
    \]
    Тогда такая система имеет решения
    \begin{gather*}
        C'_1 = u_1(x),\; \dots,\; C_n = u_n(x)\\
        y = \left[\left(\phi_1(x)\, \int u_1(x)\,dx \right) + \dots + \left( \phi_n(x)\, \int u_n(x)\,dx \right) \right] + \left[C_1\,\phi_1(x) + \dots + C_n\,\phi_n(x)\right]
    \end{gather*}
    где $C_1,\; \dots,\; C_n$ --- константы.\\
    \textcolor{cyan}{Примечание}. Тут две суммы: одна с интегралом, другая без
\end{Th}

\begin{Proof}
    Рассмотрим уравнение 
    \[
        y = C_1\,\phi_1 + \dots + C_n\,\phi_n
    \]
    Возьмём $n$ производных, причём заметим, что $\phi$, и $C_1$ зависимы от $x$. Тогда получаем
    \begin{align*}
        y &= C_1\,\phi_1 + \dots + C_n\,\phi_n\\
        y' &= C_1\,\phi'_1 + \dots + C_n\,\phi'_n +  \underbrace{C'_1\,\phi_1 + \dots + C'_n\,\phi_n}_{\equiv 0}\\
        y'' &= C_1\,\phi''_1 + \dots + C_n\,\phi''_n + \underbrace{C'_1\,\phi'_1 + \dots + C'_n\,\phi'_n}_{\equiv 0}\\
        &\vdots\\
        y^{(n-2)} &= C_1\,\phi_1^{(n-2)} + \dots + C_n\,\phi_n^{(n-2)} + \underbrace{C'_1\,\phi^{(n-3)}_1 + \dots + C'_n\,\phi^{(n-3)}_n}_{\equiv 0}\\
        y^{(n-1)} &= C_1\,\phi_1^{(n-1)} + \dots + C_n\,\phi_n^{(n-1)} + \underbrace{C'_1\,\phi^{(n-2)}_1 + \dots + C'_n\,\phi^{(n-2)}_n}_{\equiv f(x)}
    \end{align*}
    Таким образом
    \begin{align*}
        p_n\,*\; |\; &y = C_1\,\phi_1 + \dots + C_n\,\phi_n\\
        p_{n-1}\,*\; |\; &y' = C_1\,\phi'_1 + \dots + C_n\,\phi'_n\\
        p_{n-2}\,*\; |\; &y'' = C_1\,\phi''_1 + \dots + C_n\,\phi''_n\\
         & \vdots\\
        p_1\,*\; |\; &y^{(n-2)} = C_1\,\phi_1^{(n-2)} + \dots + C_n\,\phi_n^{(n-2)}\\
        1\,*\; |\; &y^{(n-1)} = C_1\,\phi_1^{(n-1)} + \dots + C_n\,\phi_n^{(n-1)} +  f(x)
    \end{align*}
    Сложим все функции, получим
    \[
        L(y) = C_1\,\underbrace{L(\phi_1)}_{\equiv 0 (\text{ФСР})} + \dots + C_n\,\underbrace{L(\phi_n)}_{\equiv 0 (\text{ФСР})} + f(x) 
    \]
    Таким образом $L(y) = f(x)$. Значит уравнение в самом начале является решением ЛНДУ.\\
    
    Теперь рассмотрим
    \[
        \begin{cases}
            C'_1\,\phi_1 + \dots + C'_n\,\phi_n = 0\\
            C'_1\,\phi'_1 + \dots + C'_n\,\phi'_n = 0\\
            \vdots\\
            C'_1\,\phi^{(n-3)}_1 + \dots + C'_n\,\phi^{(n-3)}_n = 0\\
            C'_1\,\phi^{(n-2)}_1 + \dots + C'_n\,\phi^{(n-2)}_n = f(x)
        \end{cases}
    \]
    Это система из $n$ алгебраических уравнений относительно $n$ переменных $C'_1,\, \dots,\, C'_n$. Видим, что определитель матрицы $\Delta = W(x) \neq 0$\\ $\forall x \in (a;\,b)$ (Т.к. $\{\phi_1,\; \dots,\; \phi_n\}$ --- ФСР). Следовательно, по теореме Крамера существует единственное решение
    \[
        C'_1 = u_1(x),\; \dots,\; C'_n = u_n(x)
    \]
    Таким образом
    \[
        C_1(x) = \int u_1(x)\,dx + C_1,\; \dots,\; C_n(x) = \int u_n(x)\,dx + C_n
    \]
    Значит решение ЛНДУ
    \[
        y = \left(\int u_1(x)\,dx + C_1\right)\,\phi_1(x) + \dots + \left(\int u_n(x)\,dx + C_n\right)\,\phi_n(x)
    \]
    Что аналогично
    \[
        y = \left[\left(\phi_1(x)\, \int u_1(x)\,dx \right) + \dots + \left( \phi_n(x)\, \int u_n(x)\,dx \right) \right] + \left[C_1\,\phi_1(x) + \dots + C_n\,\phi_n(x)\right]
    \]
\end{Proof}

\begin{Note}[Нахождение произвольных постоянных и описание ответа]
    Пользуясь методом Крамера $x_k = \frac{\Delta_k}{\Delta}$ получаем
    \begin{gather*}
        u_k(x) = \frac{1}{W(x)} \, 
        \begin{bmatrix}
            \phi_1 & \dots & 0 & \dots & \phi_n\\
            \phi'_1 & \dots & 0 & \dots & \phi'_n\\
            \vdots & \dots & \vdots & \dots & \vdots\\
            \phi^{(n-1)}_1 & \dots & f(x) & \dots & \phi^{(n-1)}_n
        \end{bmatrix} = \\
        = (-1)^{(k+n)}\, \frac{W(\phi_1,\; \dots,\; \phi_{k - 1}, \phi_{k + 1}\, \dots,\, \phi_n)}{W(x)}\,f(x) = \\
        = \frac{(-1)^{(k+n)}\, f(x)\, W_k(x)}{W(x)}
    \end{gather*}
    Для описания решения введём вспомогательную функцию
    \[
        G(x,\, t) = \sum_{k=1}^{n} (-1)^{k + n} \phi_k(x)\, \frac{W_k(t)}{W(t)}
    \]
    \textcolor{cyan}{Примечание.} Переменная $t$ введена, чтобы <<разграничить>> интегрируемые Вронскианы и $\phi(x)$, который за интегралом.
    Тогда решение имеет вид
    \[
        y = \underbrace{\int_{x_0}^{x}G(x,\,t)\, f(t)\, dt}_{y_\text{ч.н.}} + \underbrace{\sum_{k=1}^{n} C_k\, \phi_k(x)}_{y_\text{о.о.}}
    \]
\end{Note}

\begin{Example}
    Дано
    \[
        x\,y'' + y' = x
    \]
    Решение.\\
    Приведём к каноническому виду (делим на $x$)
    \[
        y'' + \frac{y'}{x} = 1
    \]
    Получили $L(y)$. Решаем в два этапа
    \begin{enumerate}
        \item Рассматриваем однородное уравнение $L(y) = 0$
        \begin{gather*}
            y'' + \frac{y'}{x} = 0\\
            y' = z, \qquad y'' = z'\\
            z' + \frac{z}{x} = 0\\
            \int \frac{dz}{z} = - \int \frac{dx}{x}\\
            \ln|z| = - \ln|x| + \ln\,C_1\\
            z = y' = \frac{C_1}{x}\\
            y = y_{\text{о.о.}} = C_1\, \ln|x| + C_2
        \end{gather*}
        
        \item Решаем неоднородное уравнение с произволными постоянными $C_1(x),\; C_2(x)$ и $\phi_1,\; \phi_2$
        \begin{gather*}
            y = C_1(x)\,\ln|x| + C_2(x)\\
            \phi_1 = \ln|x|, \qquad \phi_2 = 1\\
            \phi'_1 = \frac{1}{x}, \qquad \phi'_2 = 0
        \end{gather*} 
        Накладываем 1 дополнительное условие $C'_1\,\ln|x| + C'_2 = 0$\\
        \[
            \begin{cases}
                C'_1\,\ln|x| + C'_2 = 0\\
                C'_1\,\frac{1}{x} + C'_2\,0 = 1 \quad (\text{Из условия})
            \end{cases}
        \]
        Решаем
        \begin{gather*}
            \begin{cases}  
                C'_1\,ln|x| + C'_2 = 0\\
                C'_1\,\frac{1}{x} = 1
            \end{cases}\\
            \begin{cases}  
            C'_2 = - C'_1\,ln|x|\\
            C'_1 = x
            \end{cases}\\
            \begin{cases}  
            C'_2 = - x\,ln|x|\\
            C_1(x) = \frac{x^2}{2} + C_1
            \end{cases}\\
            \begin{cases}  
            C_2(x) = -\int x\,ln|x|\,dx\\
            C_1(x) = \frac{x^2}{2} + C_1
            \end{cases}\\
            C_2(x) = -\int x\,ln|x|\,dx = \left[u = ln|x|, \quad dv = x\,dx; \qquad du = \frac{1}{x}, \quad v = \frac{x^2}{2}\right] = \\
            = - \frac{x^2}{2}\,ln|x| + \frac{x^2}{4} + C_2
        \end{gather*}
        Зная значения $C_1(x)$ и $C_2(x)$ получаем
        \[
            y = \frac{x^2}{2}\,ln|x| - \frac{x^2}{2}\,ln|x| + \frac{x^2}{4} + C_1\,ln|x| + C_2
        \]
        Ответ
        \[
            y = \frac{x^2}{4} + C_1\,ln|x| + C_2
        \]
    \end{enumerate}
\end{Example}