\author{Tkachuk Andrei}

\section{Системы обыкновенных ДУ}

\begin{Def}
    Система уравнений вида
    \[
        F_i(x,\; y_1,\; y_1',\; \dots,\; y_1^{m_1},\; \dots,\; y_n,\; y_n',\; \dots,\; y_1^{m_n}) \equiv 0 \qquad (i = 1,\; 2,\; \dots,\; n)
    \]
    называют системой обыкновенных ДУ порядка $m = \max\{m_1,\; \dots,\; m_n\}$ относительно функций 
    $y_1 = y_1(x),\; \dots,\; y_n = y_n(x)$. Набор функций $\{y_1,\; \dots,\; y_n\}$ называется решением ДУ если при подстановке 
    \[
        F_i(x,\; y_1,\; y_1',\; \dots,\; y_1^{m_1},\; \dots,\; y_n,\; y_n',\; \dots,\; y_1^{m_n}) \stackrel{(x)}{\equiv} 0 \qquad (i = 1,\; 2,\; \dots,\; n)
    \]
    то есть получается система верных функциональных тождеств 
\end{Def}

\begin{Th}
    Всякая система ДУ из определения 1 эквивалентна некоторой системе обыкновенных ДУ 1-го порядка
\end{Th}

\begin{Proof}
    Введём
    \begin{align*}
        & z_{1,1} = y_1 && z_{1,2} = y'_1 && \dots && z_{1,m_1} = y^{(m_1-1)}_1 \\
        &\vdots && \vdots && \dots && \vdots \\
        & z_{n,1} = y_n && z_{n,2} = y'_n && \dots && z_{n,m_n} = y^{(m_n-1)}_n
    \end{align*}
    Примечание. Всего получим $N = \sum_{i=1}^{n} m$ функций.\\
    Таким образом, после замены на $z$ получаем
    \begin{multline*}
        F_i(x,\; z_{1,1},\; z_{1,2},\; \dots,\; z_{1,m_1},\; z'_{1,m_1},\; \dots,\; z_{n,1},\; z_{n,2},\; \dots,\; z_{n,m_n},\; z'_{n,m_n}) \equiv 0\\
        (i = 1,\; 2,\; \dots,\; n)
    \end{multline*}
    Мы знаем, что $z_{1,1},\; z_{1,2}$ связаны межу собой и другими $z_{1,i}$, тогда
    \[
        \begin{cases}
            z'_{i,j} = z'_{i,j + 1} \qquad 1\leqslant i \leqslant n \quad 1\leqslant j \leqslant m_i - 1\\
            F_i = 0
        \end{cases}
    \]
    Таким образом мы получили систему обыкновенных ДУ первого порядка относительно $z_{i,j}$ 
\end{Proof}

\begin{Def}
    Сустема ДУ вида
    \[
        \begin{cases}
            y'_1 = f_1(x,\; y_1,\; \dots,\; y_n)\\
            \dots\\
            y'_n = f_n(x,\; y_1,\; \dots,\; y_n)
        \end{cases}
    \]
    Называют нормальной системой обыкновенных ДУ 1-го порядка
\end{Def}

\begin{Note}
    Введём 
    \[
        \vec{y} = \begin{pmatrix} y_1\\ \vdots\\ y_n \end{pmatrix} \quad \vec{y'} = \begin{pmatrix} y'_1\\ \vdots\\ y'_n \end{pmatrix} \quad \vec{f} = \begin{pmatrix} f_1\\ \vdots\\ f_n \end{pmatrix}
    \]
    Тогда систему из определения выше можно описать так
    \[
        \vec{y'} = \vec{f}(x,\; \vec{y})
    \]
\end{Note}

\begin{Example}
    Дано
    \[
        y^{(n)} + p_1(x)\,y^{(n-1)} + p_2(x)\,y^{(n-2)} + \dots + + p_{n-1}\,y' + p_n(x)\,y = f(x)
    \]
    Требуется перейти к нормальной ситеме обыкновенных ДУ 1-го порядка.\\
    Решение.\\
    Введём следующие функции
    \[
        y_1 = y, \quad y_2 = y', \quad \dots, \quad y_n = y^{(n-1)}
    \]
    Тогда ответ
    \[
        \begin{cases}
            y_n' = -p_n\,y_1 - p_{n-1}\,y_2 - \dots - p_1\,y_n + f(x)\\
            y'_{n-1} = \frac{-p_n\,y_1 - p_{n-1}\,y_2 - \dots - p_2\,y_{n-1} + f(x)}{p_1(x)} - y'_n\\
            \dots\\
            y'_1 = \frac{- y_1\,p_n(x) + f(x)}{p_{n-1}} - y'_2
        \end{cases}
    \]
\end{Example}

\begin{Th}
    Пусть 
    \[
        \begin{cases}
            y'_1 = f_1(x,\; y_1,\; \dots,\; y_n)\\
            \vdots\\
            y'_n = f_n(x,\; y_1,\; \dots,\; y_n)\\
        \end{cases}
    \]
    --- нормальная система относительно $n$ фукций $y_1,\; \dots,\; y_n$ первого порядка\\
    
    Тогда если функции
    \[
        f_1,\; \dots,\; f_n,\; \frac{\delta f_i}{\delta y_j} \qquad 1 \leqslant i \leqslant n \quad 1\leqslant j \leqslant n
    \]
    непрерывны в области $D \subset \bb{R}^{n+1}$, а $M_0(x_0,\; y^0_1,\; \dots,\; y^0_n)$ --- внутренняя точка области $D$, то $\exists \delta > 0$ и существует единственное решение 
    \[
        (y_1(x),\; \dots,\; y_n(x)) \qquad x \in U_\delta(x_0)
    \] 
    удовлетворяющее следующим условиям задачи Коши
    \[
        \begin{cases}
            \vec{y'} = \vec{f'}(x,\; \vec{y})\\
            \vec{y}|_{x = x_0} = \vec{y_0} = 
            \begin{pmatrix}
                y^0_1\\
                \vdots\\
                y^0_n
            \end{pmatrix}
        \end{cases}
    \]
\end{Th} 
% можно добавить прим. из конспекта
\begin{Proof}
    Без доказательств
\end{Proof}

\begin{Note}[Как построить инетгральную кривую]
    Выведем систему ДУ <<в дифференциалах>>. Для этого рассмотрим
    \[
        \begin{cases}
            y'_1 = f_1(x,\; y_1,\; \dots,\; y_n)\\
            \vdots\\
            y'_n = f_n(x,\; y_1,\; \dots,\; y_n)\\
        \end{cases}
    \]
    в сокращенном виде
    \[
        \begin{cases}
            y'_1 = f_1(x, \vec{y})\\
            \vdots\\
            y'_n = f_n(x, \vec{y})\\
        \end{cases}
    \]
    это равносильно
    \[
        dx = \frac{dy_1}{f_1(x, \vec{y})},\; \dots,\; \frac{dy_n}{f_n(x, \vec{y})}
    \]
    Воспользуемся $dy_i = y'_i(x)\,dx$, получим
    \[
        1 = \frac{y'_1}{f_1(x, \vec{y})},\; \dots,\; \frac{y'_n}{f_n(x, \vec{y})}
    \]
    
    Если интегральная кривая $L$ проходит через точку $M_0(x_0,\; y^0_1,\; \dots,\; y^0_n)$ то касательные к ней удовлетворяют системе линеных алгебраических уравнений
    \[
        X - x_0 = \frac{Y_1 - y^0_1}{f_1(M_0)} = \dots = \frac{Y_n - y^0_n}{f_n(M_0)}
    \]
\end{Note}

\begin{Note}
    Систему ДУ можно свести к уравнению высокого порядка относительно одной функции с помощью метода исключения неизвестной.
\end{Note}

\begin{Example}
    Дано
    \[
        \begin{cases}
            y' = x + z\\
            z' = x + 2\,y + z
        \end{cases} \qquad \text{Найти } (y(x),\; z(x))
    \]
    Решение.\\
    Продиференцируем первое уравнение по $x$, а также выразим из него $z$, получим
    \[
        \begin{cases}
            y'' = 1 + z' = [z' = x + 2\,y + z] = 1 + x + 2\,y + z\\
            z = y' - x
        \end{cases}
    \]
    Заменяем (и исключаем) $z$ получаем
    \begin{gather*}
        y'' = 1 + x + 2\,y + y' - x\\
        y'' - y' - 2\,y = 1\\
        P(\lambda) = \lambda^2 - \lambda - 2 = (\lambda + 1)\,(\lambda - 2)\\
        \alpha_1 = -1 \quad \alpha_2 = 2 \quad k_{1, 2} = 1
    \end{gather*}
    Получаем ФСР $\{e^{-x}\,\; e^{2\,x}\}$ и тогда,
    \begin{gather*}
        y_{\text{о.о.}} = C_1\,e^{-x} + C_2\,e^{2x}\\
        \alpha = 0,\quad k = 0,\quad m=0
    \end{gather*}  
\end{Example}