\section{Равномерно сходящиеся функциональные ряды}

\newenvironment{rcases}
{\left.\begin{aligned}}
	{\end{aligned}\right|}

\begin{Def}
	Пусть ${u_n(x)}$ - функциональная последовательность, определённая на $D,  x \in D$\\
	Тогда $\baserow{u_n(x)}$ - функциональный ряд $x \in D$. Этот ряд сходится $\Leftrightarrow\\
	\Leftrightarrow \exists \lim\limits_{n \to +\infty}S_n(x) = S(x), x \in D$. При этом: $S(x) = \baserow{u_n(x)}$ - сумма ряда ($x \in D$). Т.е. ряд сходится к сумме $S$ на $D \Leftrightarrow S_n(x) \to S(x) (n \to +\infty; x \in D)$
\end{Def}

\begin{Def}
	В определении 1 говорят, что ряд равномерно сходится на области $D, D \subset \bb{R} \Leftrightarrow S_n(x) \rightrightarrows S(x) (n \to +\infty, x \in D)$
\end{Def}

\begin{Th}(Критерий Коши равномерной сходимости функционального ряда)\\
	Пусть $\baserow{u_n(x)}, x \in D, D \subset \bb{R}$. Тогда этот ряд равномерно сходитсян на $D \Leftrightarrow\\
	\Leftrightarrow \forall \varepsilon > 0 \; \exists N(\varepsilon) \in \bb{N} \; \forall n \geq N \; \forall p \geq 1 \; \forall x \in D \: |u_{n+1}(x)+\dots+u_{n+p}(x)| < \varepsilon$ 
\end{Th}

\begin{Proof}
	$S_n(x) = \sum\limits_{k=1}^{n}u_n(x)$ - частичная сумма ряда, тогда $\baserow{u_n(x)}$ равномерно сходится на $D \Leftrightarrow S_n(x) \rightrightarrows S(x) (n \to +\infty; x \in D) \Leftrightarrow\\
	\Leftrightarrow \text{[критерий Коши для функциональных последовательностей]} \Leftrightarrow \\
	\Leftrightarrow \forall \varepsilon > 0 \; \exists N(\varepsilon) \in \bb{N} \; \forall n \geq N \; \forall p \geq 1 \; \forall x \in D \: |u_{n+1}(x)+\dots+u_{n+p}(x)| < \varepsilon$
\end{Proof}

\begin{Th}(О непрерывности суммы равномерно сходящегося функционального ряда)\\
	Пусть $\forall n \geq 1 \: u_n(x) \in C_{[a;b]} \text{ и ряд } \baserow{u_n(x)} = S(x)$ равномерно сходится к сумме $S(x) \text{ на } [a;b]$. Тогда $S(x) \in C_{[a;b]}$
\end{Th}

\begin{Proof}
	Рассмотрим частичные суммы $S_n(x) = \sum\limits_{k=1}^{n} u_k(x) = u_1(x) (\in C_{[a;b]}) + \dots + u_n(x)(\in C_{[a;b]}) \in C_{[a;b]}\\$
	$S_n(x) \rightrightarrows \S(x) (n \to +\infty; x \in [a;b]) \Rightarrow \text{[по теореме 2 предыдущего параграфа]} \Rightarrow S(x) \in C_{[a;b]}$
\end{Proof}

\begin{Th}(О почленном интегрировании равномерно сходящихся рядов)\\
	Пусть $\forall n \geq 1, u_n(x) \in C_{[a;b]}$ и ряд $\baserow{u_n(x)} = S(x)$ равномерно сходится на $[a;b]$\\
	Тогда $\int\limits_{a}^{b}S(x)dx = \int\limits_{a}^{b}\baserow{u_n(x)}dx = \baserow{\int\limits_{a}^{b}u_n(x)}$
\end{Th}

\begin{Proof}~\\
	$S_n(x) \rightrightarrows S(x) (n \to +\infty; x \in [a;b]) \forall n \geq 1 \: s_n(x) in C_{[a;b]} \Rightarrow\\
	\Rightarrow \text{по теореме 2} \Rightarrow S(x) \in C_{[a;b]}$ (т.е. все интегралы существуют)\\
	По теореме 3 предыдущего параграфа $\int\limits_{a}^{b}S_n(x) \rightrightarrows \int\limits_{a}^{b}S(x) (n \to +\infty) \Rightarrow\\
	\Rightarrow \int\limits_{a}^{b}S_n(x) = \int\limits_{a}^{b}(\sum\limits_{k=1}^{n}u_k(x))dx = \sum\limits_{k=1}^{n}\int\limits_{a}^{b}u_k(x)dx$ - частичная сумма ряда $\baserow{\int\limits_{a}^{b}u_n(x)dx}$. Она будет стремиться к сумме $\baserow{\int\limits_{a}^{b}u_n(x)dx} = \int\limits_{a}^{b}S(x)dx$
\end{Proof}

\begin{Note}
	$\forall x \in [a;b]$ в условиях теоремы 3 $\baserow{\int\limits_{a}^{x} u_k(t)dt} = \int\limits_{a}^{x}S(t)dt$
\end{Note}

\begin{Th}(О почленном дифференциировании равномерно сходящихся функциональных рядов)
	Пусть $\forall n \geq 1$ функции $u_n(x) \in C_{[a;b]}^{1}$ и выполняются следущие условия:\\
	\begin{enumerate}
		\item $\baserow{u_n'(x)} \rightrightarrows \phi(x)$ равномерно сходится на $[a;b]$
		\item $\exists x_0 \in [a;b] \: \baserow{u_n(x_0)} = A, A \in \bb{R}$ 
	\end{enumerate}
	Тогда:
	\begin{enumerate}
		\item $\baserow{u_n(x)}$ равномерно сходится к функции $S(x) \in C_{[a;b]}^{1} \text{на} [a;b]$
		\item $S'(x) = \phi(x), x \in [a;b]$, в частности: \begin{itemize}
			\item $S'(a+0) = \phi(a)$
			\item $S'(b-0) = \phi(b)$
		\end{itemize}
	\end{enumerate}
\end{Th}

\begin{Proof}
	$\sum\limits_{k=1}^{n}u_k'(x) \rightrightarrows \phi(x) (n \to +\infty x \in [a;b]) \Rightarrow\\
	\Rightarrow$ 
	$\begin{rcases}
		(\sum\limits_{k=1}^{n}u_k(x))' \rightrightarrows \phi(x) (n \to +\infty x \in [a;b])\\
		\sum\limits_{k=1}^{n}u_k(x_0) \to A (n \to +\infty)
	\end{rcases} \Rightarrow \text{[по теореме 4 \S 1]} \Rightarrow\\
	\Rightarrow
	\begin{cases}
		\sum\limits_{k=1}^{n}u_k(x) \rightrightarrows S(x) (n \to +\infty x \in [a;b]) \quad (1)\\
		S'(x) = \phi(x), x \in [a;b]
	\end{cases}\\
	\phi(x) \in C_{[a;b]} \Rightarrow S(x) \in C_{[a;b]}^{1}$\\
	Из $(1) \Rightarrow \baserow{u_n'(x)} = S(x)$ равномерно сходится на отрезке $[a;b]$
\end{Proof}

\begin{Th}(Достаточный признак Вейерштрасса равномерной сходимости функциональных рядов)
	Пусть $\baserow{u_n(x)}$ - функциональный ряд для $x \in D, D \subset \bb{R}$\\
	Тогда, если найдется числовой ряд $\exists \baserow{a_n}$, который удовлетворяет:
	\begin{enumerate}
		\item $\forall x \in D \; \forall n \in \bb{N} \quad |u_n(x)| \leq a_n$
		\item $\baserow{a_n}$ сходится
	\end{enumerate}
	Тогда ряд $\baserow{u_n(x)}$ равномерно сходится на $D$
\end{Th}

\begin{Proof}
	Ряд $\baserow{a_n}$ сходится $\Rightarrow$ [по критерию Коши] $\Rightarrow\\
	\Rightarrow \forall \varepsilon > 0 \; \exists N(\varepsilon) \in \bb{N} \; \forall n \geq N \; \forall p \geq 1 \quad |a_{n+1} + \dots + a_{n+p}| < \varepsilon$\\
	$0 \leq u_n(x) \leq a_n \Rightarrow \forall n \geq 1 \quad a_n \geq 0;\\
	\forall x \in D \quad |u_{n+1}(x) + \dots + u_{n+p}(x)| \leq |u_{n+1}(x)| + \dots + |u_{n+p}(x)| \leq \\
	\leq a_{n+1} + \dots + a_{n+p} = |a_{n+1} + \dots + a_{n+p}| < \varepsilon \Rightarrow\\
	\Rightarrow \forall \varepsilon > 0 \; \exists N(\varepsilon) \in \bb{N} \; \forall n \geq N \; \forall p \geq 1 \quad |u_{n+1}(x) + \dots + u_{n+p}(x)| < \varepsilon \Rightarrow\\
	\Rightarrow \text{[по теореме 1 \S]} \Rightarrow$ ряд $\baserow{u_n(x)}$ равномерно сходится на $D$
\end{Proof}